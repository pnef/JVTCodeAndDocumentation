The suppression of pileup jets has been a crucial component of many physics analyses
using 2012 LHC proton-proton collisions. In ATLAS, tracking information has been used to
calculate a variable called the jet-vertex-fraction, which is the fraction of the total momentum 
of tracks in the jet which is associated to the primary vertex. Imposing a minimum on
this variable rejects the majority of pileup jets, but leads to hard-scatter jet efficiencies that
depend on the number of reconstructed primary vertices in the event (NVtx). In this note,
new track-based variables to suppress pileup jets are developed in such a way that the resulting 
hard-scatter jet efficiency is stable as a function of NVtx. A multivariate combination of
two such variables called the jet-vertex-tagger (JVT) is constructed. The modeling of JVT
is tested in Z -> μμ+jets as well as in semileptonic ttbar events. The efficiency of different minimal 
JVT criteria are measured in data and compared to simulation. In addition, it is shown
that jet-vertex association can be applied to large-R jets, providing a track-based grooming
technique that is as powerful as calorimeter-based trimming but based on complementary
tracking information. Finally, the performance of track-based grooming is compared with
the recently proposed jet cleansing algorithm.
